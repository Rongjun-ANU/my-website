% LaTeX resume using res.cls
\documentclass{res}
%\usepackage{helvetica} % uses helvetica postscript font (download helvetica.sty)
%
% Don't adjust any of these commands
%
%\usepackage{newcent}   % uses new century schoolbook postscript font  
\setlength{\topmargin}{-0.7in}  % Start text higher on the page 
\setlength{\textheight}{9.8in}  % increase textheight to fit more on a page
\setlength{\headsep
}{0.in}     % space between header and text
\setlength{\headheight}{11pt}   % make room for header
\usepackage{fancyhdr}  % use fancyhdr package to get 2-line header
\renewcommand{\headrulewidth}{0pt} % suppress line drawn by default by fancyhdr
%\rhead{\hfill- \thepage~-\hfill}  % put page number at center
\cfoot{\hfill- \thepage~-\hfill}  % the footer is empty
\pagestyle{fancy} % set pagestyle for the document
\setlength{\parindent}{0cm}
\newcommand{\HRule}{\rule{\linewidth}{0.5mm}}

\lfoot{\hspace*{-\sectionwidth}R. Huang} % force lhead all the way left

\usepackage[hidelinks]{hyperref}
\usepackage{orcidlink}
\usepackage{xcolor}
\usepackage{xeCJK}
\setCJKmainfont{Noto Serif CJK SC}

\begin{document} 
\thispagestyle{empty} % this page does not have a header
\begin{center}
{\bf \LARGE
%Andrew J. Battisti\\ \vspace{5pt}
Rongjun Huang\orcidlink{0000-0002-6646-8365} {\bf \Large(黄镕钧)\\ \vspace{4pt}}}


Email: {\bf Rongjun.Huang@icrar.org}, \\
{\bf Rongjun.Huang@research.uwa.edu.au, Astro@Rongjun-Huang.com}; \\
Web: \href{https://www.Rongjun-Huang.com}{\bf www.Rongjun-Huang.com} \\
% u6569836@anu.edu.au \\ 
% Phone: +61 0403244925
\end{center}



\begin{resume}

\vspace{-0.4in} 
\section{\bf EDUCATION}
\vspace{0.1in} 
\moveleft\hoffset\vbox
{\hrule width\resumewidth height 1pt}\smallskip
\vspace{-0.1in} 
    {\bf The University of Western Australia}, The International Centre for Radio Astronomy Research, Crawley, WA, Australia 
	\begin{itemize} 
      \item[] 
        PhD student - {\bf The Doctor of Philosophy}, 2025-2029. 
      \item[]
        \textit{Thesis: Environmental effects on Spatially-resolved Star Formation and Interstellar Medium Properties in the Virgo Cluster.} \\ \hfill Supervisor: Prof. Luca Cortese
        %Committee:
  \end{itemize}

    {\bf The Australian National University}, Research School of Astronomy \& Astrophysics, Weston Creek, ACT, Australia 
	\begin{itemize} 
      \item[] 
        M.Sc - {\bf Master of Science (Advanced) - Astronomy and Astrophysics}, GPA - 7/7 With {\bf Commendation}, June 2024. 
      \item[]
        \textit{Thesis: Negative Metallicity Gradient in Galactic Outflows Based on QED Simulations.} \\ \hfill Supervisor: Prof. Mark Krumholz
        %Committee:
  \end{itemize}

    {\bf The Australian National University}, Research School of Astronomy \& Astrophysics, Weston Creek, ACT, Australia 
	\begin
{itemize} 
      \item[] 
        H.Sc - {\bf Bachelor of Science (Honours) - Astronomy and Astrophysics}, {\bf First Class Honours} (H1, 82/100), December 2022. 
      \item[]
        \textit{Thesis: Exploring the Intrinsic Scatter of the Star-Forming Galaxy Main Sequence at Redshift 0.5 to 3.0.} \\ \hfill Supervisor: Dr. Andrew Battisti
        %Committee:
  \end{itemize}
  
    {\bf
 The Australian National University}, Research School of Astronomy \& Astrophysics, Acton, ACT, Australia 
	\begin{itemize} 
      \item[]
        B.Sc - {\bf Bachelor of Science - Astronomy and Astrophysics}, GPA - 5.6/7, June 2021.  
      \item[]
        \textit{Thesis: Using \texttt{MAGPHYS+photo-z} to Characterise the Properties of Star-Forming Galaxies.} \\ \hfill Supervisor: Dr. Andrew Battisti
    %Committee:
  \end{itemize}
 

%\vspace{-0.2in} 

%\vspace{0.05in} 
% \section{\bf RESEARCH EXPERIENCES} 
% \vspace{0.1in}
% \moveleft\hoffset\vbox{\hrule width\resumewidth height 1pt}\smallskip
% \vspace
% {-0.1in} 

%   {\bf ASTR8010 Master Research}  (07/23--05/24; HD, 81/100)  The Australian National University. \\ \hfill Supervisor: Prof. Mark Krumholz
% 	\begin{itemize} 
%       \item[] Negative Metallicity Gradient in Galactic Outflows Based on QED Simulations. 
%   \end{itemize} 

%   {\bf ASTR4005 Honours Research}  (02/22--10/22; HD, 86/100)  The Australian National University. \\ \hfill Supervisor: Dr. Andrew Battisti
% 	\begin{itemize} 
%       \item[] Exploring the Intrinsic Scatter of the Star-Forming Galaxy Main Sequence at Redshift 0.5 to 3.0. \\
%       Nominated for the 2023 Bok Prize of the Astronomical Society of Australia. 
%   \end{itemize}  

%   {\bf ASTR3005 Summer Research}  (12/20--02/21; D, 77/100)  The Australian National University. \hfill \\ Supervisor: Dr. Andrew Battisti
% 	\begin{itemize} 
%       \item[] Using \texttt{MAGPHYS+photo-z} to Characterise the Properties of Star-Forming Galaxies. 
%   \end{itemize}  
  
% \vspace{0.05in} 

%\section{\bf PEER-REVIEWED PUBLICATIONS}
\section{\bf FIRST-AUTHOR PUBLICATIONS}
\vspace{
0.1in}
\moveleft\hoffset\vbox{\hrule width\resumewidth height 1pt}\smallskip
%\vspace{-0.1in} 



\begin{enumerate}

    \item 
    {\bf \textit{MAUVE--MUSE: The Origin of Spatially-resolved Mass--Metallicity Relation's Secondary Dependence on Star Formation Rate Surface Density}}  
	\begin{itemize} 
      \item[] {\bf Rongjun Huang}, Luca Cortese, MAUVE--Collaboration. \textbf{2026}. \\
      (In prep).
  \end{itemize}

    \item 
    {\bf \textit{\textsc{Quokka}-based understanding of outflows (QED) - IV. Limitations of H$\alpha$ as an outflow diagnostic}}  
	\begin{itemize} 
      \item[] {\bf Rongjun Huang}, Aditi Vijayan, Mark R. Krumholz. \textbf{2026}. \\
      MNRAS (Submitted).  \href{https://doi.org/10.48550/arXiv.2511.05056}{\bf DOI: 10.48550/arXiv.2511.05056}.
  \end{itemize}

    \item 
    {\bf \textit{\textsc{Quokka}-based understanding of outflows (QED) - II. X-ray metallicity gradients as a signature of galactic wind metal loading}}  
	\begin{itemize} 
      \item[] {\bf Rongjun Huang}, Aditi Vijayan, Mark R. Krumholz. \textbf{2025}. \\
      MNRAS (Published, Volume 539, Issue 2, May 2025, Pages 1723–1737).  \href{https://doi.org/10.1093/mnras/staf593}{\bf DOI: 10.1093/mnras/staf593}.  Citations: 1 (per NASA/ADS)
      % \href{https://doi.org/10.1093/mnras/stad108}{\bf DOI: 10.1093/mnras/stad108}.  
    
  \end{itemize}
  
    \item 
    {\bf \textit{Exploring the Intrinsic Scatter of the Star-Forming Galaxy Main Sequence at Redshift 0.5 to 3.0}}  
	\begin{itemize} 
      \item[] {\bf Rongjun Huang}, Andrew J. Battisti, Kathryn Grasha, Elisabete da Cunha, Claudia del P Lagos, Sarah K. Leslie \& Emily Wisnioski. \textbf{2023}. \\
      MNRAS (Published, Volume 520, Issue 1, March 2023, Pages 446–460).  \href{https://doi.org/10.1093/mnras/stad108}{\bf DOI: 10.1093/mnras/stad108}. Citations: 11 (per NASA/ADS) 
    %   \item[] \emph{Rongjun Huang}, Andrew J. Battisti, Kathryn Grasha, Elisabete da Cunha, Claudia del P Lagos, Sarah K. Leslie \& Emily Wisnioski. \\
    %   MNRAS (Published). arXiv identifier: \href{https://arxiv.org/abs/2301.01995}{2301.01995}.

  \end{itemize}

\end{enumerate}

\section{\bf GRANTED OBSERVATION}
\vspace{0.1in}
\moveleft\hoffset\vbox{\hrule width\resumewidth height 1pt}\smallskip
\vspace{-0.1in} 

{\bf Anatomy of a fall: Dissecting the environment-driven transformation of late-type Virgo cluster galaxies with HST UV-optical imaging of star clusters, associations, and HII regions}
	\begin{itemize} 
      \item[] Co-PI -- Hubble Space Telescope (HST) -- 145 orbits ($\sim$108.8 hr). \\
  \end{itemize}

\section{\bf OBSERVATION EXPERIENCE}
\vspace{0.1in}
\moveleft\hoffset
\vbox{\hrule width\resumewidth height 1pt}\smallskip
% \vspace{-0.1in} 

{\bf Extremely Metal Poor Stars in Milky Way}
	\begin{itemize} 
      \item[] ANU 2.3m Telescope, Siding Spring Observatory (SSO), NSW, Australia \\
      \hfill September 2019; 3 nights
  \end{itemize}  

{\bf The GALactic Archaeology with HERMES (GALAH) survey in 2024A}
	\begin{itemize} 
      \item[] Anglo-Australian Telescope (AAT), Siding Spring Observatory (SSO), NSW, Australia \\
      \hfill July 2024 [remote observing through Mount Stromlo Observatory (MSO) observation room]
  \end{itemize} 

\section{\bf HPC EXPERIENCE}
\vspace{0.1in}
\moveleft\hoffset\vbox{\hrule width\resumewidth height 1pt}\smallskip
\vspace{-0.1in} 

{\bf NCI / Gadi}
	\begin{itemize}
      \item[] Project jh2: Star Formation and Feedback in a Turbulent Interstellar Medium. 2023-present.
  \end{itemize}

{\bf Pawsey / Setonix}
	\begin{itemize}
      \item[] Pawsey0807. 2023-present.
  \end{itemize}

{\bf Swinburne / OzSTAR}
	\begin{itemize}
      \item[] IFS Data Analysis (oz084). 2025-present.
  \end{itemize}

{\bf CADC / CANFAR}
	\begin{itemize}
      \item[] Multiphase Astrophysics to Unveil the Virgo Environment (MAUVE). 2025-present.  
  \end{itemize}

\section{\bf SOFTWARE DEVELOPMENT}
\vspace{0.1in}
\moveleft\hoffset\vbox{\hrule width\resumewidth height 1pt}\smallskip
\vspace{-0.1in} 

{\bf The nGIST Pipeline: A galaxy IFS analysis pipeline for modern IFS data}
	\begin{itemize} 
      \item[] \href{https://github.com/geckos-survey/ngist}{https://github.com/geckos-survey/ngist}
  \end{itemize}


{\bf Post-processing pipeline for \textsc{Quokka} simulation code as part of the \texttt{yt} frontend}
	\begin{itemize} 
      \item[] \href{https://github.com/chongchonghe/yt/tree/Rongjun-ANUquokka-frontend}{https://github.com/chongchonghe/yt/tree/Rongjun-ANUquokka-frontend}
  \end{itemize}


% \end{enumerate}

\section{\bf INTERNSHIP}
\vspace{0.1in}
\moveleft\hoffset\vbox{\hrule width\resumewidth height 1pt}\smallskip
\vspace{-0.1in} 
{\bf The Stability of the WiFeS Instrument on the ANU 2.3-Metre Robotic Telescope} \\ Supervisor: A/Prof. Chris Lidman \\
    \begin{itemize} 
      \item[] ANU 2.3m Telescope, Siding Spring Observatory (SSO), NSW, Australia \\
      \  \hfill  01/29/2024--02/16/2024 
    %Committee:
  \end{itemize}

\section{\bf PROFESSIONAL AFFILIATIONS} 
\vspace{0.1in} 
\moveleft\hoffset\vbox{\hrule width\resumewidth height 1pt}\smallskip
\vspace{-0.1in} 

    Member (Student) -- ARC Centre of Excellence for All Sky Astrophysics in 3 Dimensions (ASTRO 3D), Australia \hfill  2020-2024

\section{\bf CONFERENCE TALKS}
\vspace{0.1in}
\moveleft\hoffset\vbox{\hrule width\resumewidth height 1pt}\smallskip
\vspace{-0.1in} 

    {\bf
 Exploring the Intrinsic Scatter of the Star-Forming Galaxy Main Sequence at Redshift 0.5 to 3.0}  
	\begin{itemize} 
      \item[] {\it R. Huang}. 2022 ASTRO 3D Science Meeting, Burnley, VIC, Australia (Talk; June 2022).  
  \end{itemize}

    {\bf Sparkler Talk (1-min): Intrinsic Scatter of the Star-Forming Galaxy Main Sequence}  
	\begin{itemize} 
      \item[] {\it R. Huang}. 2023 ASTRO 3D Science Meeting, Freemantle, WA, Australia (Talk; June 2023).  
  \end{itemize}


\section{\bf GRANTS AND AWARDS}

%\section{\bf GRANTS, FELLOWSHIPS, AND AWARDS}
\vspace{0.1in}
\moveleft\hoffset\vbox
{

\hrule width\resumewidth height 1pt}\smallskip  
\vspace{-0.1in} 

UWA Data Institute Awards 2025 Travel Grants: 2000 AUD \hfill Decemember 2025 \\
ASTRO 3D travel funds: 100 AUD \hfill June 2022 \\
ASTRO 3D travel funds: 1500 AUD \hfill June 2023 \\

\section{\bf LANGUAGE SKILLS}
%\section{\bf GRANTS, FELLOWSHIPS, AND AWARDS}
\vspace{0.1in}
\moveleft\hoffset\vbox{\hrule width\resumewidth height 1pt}\smallskip
  
\vspace{-0.1in} 

{\bf Pearson Test of English (PTE) Academic}
    \begin{itemize} 
    \item[] Overall score 81: Listening 83, Reading 74, Speaking 90, Writing 80.  \hfill Australia, 03 Jul 2024 
    \end{itemize} 

\section{\bf PROFESSIONAL SKILLS} 
\vspace{0.1in}
\moveleft\hoffset\vbox{\hrule width\resumewidth height 1pt}\smallskip
% \vspace{-0.1in} 
    \begin{itemize} 
    \item Advanced experience with large astrophysical datasets (integral-field spectroscopy, simulations) and statistical modelling.
    \item Daily user of Python (NumPy, SciPy, Astropy, Matplotlib, Jupyter), plus experience with HPC environments and automated analysis pipelines.
    \item Comfortable with Mathematica, MATLAB, IDL, and common scientific tools (Emacs, DS9, ImageJ, LaTeX/BibTeX, Markdown, office suites).
    \end{itemize}
%\vspace{0.05in} 
 

% \section{\bf SKILLS AND COMPETENCE}
% \vspace{0.1in}
% \moveleft\hoffset\vbox{\hrule width\resumewidth height 1pt}\smallskip  
% % \vspace{-0.1in} 
%     \begin{itemize} 
%     \item Excellent problem-solving and critical thinking abilities when facing complex problems; adept in coming up with a feasible solution or a reasonable explanation based on my knowledge; 
%     \item Strong adaptability and pleasant personalities; capable of settling into a new environment very quickly and making corresponding adjustments to accommodate the change;  
%     \item Outgoing student having excellent communication skills and teamwork spirit;
%     % \item Well-disciplined and proactive research habit; accumulated abundant research experiences from different projects. 
%     \end{itemize}
% %\vspace{0.05in} 

%   \vspace{-0.1in} Caltech Submillimeter Observatory,  Mauna Kea, HI \hfill Summer 2009; 3 nights  

\end{resume}
\end{document}

